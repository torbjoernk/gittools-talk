\documentclass[ngerman,hyperref={pdfpagelabels=false}]{beamer}

\mode<handout>{
  \usepackage{pgf}
  \usepackage{pgfpages}
  \pgfpagesuselayout{2 on 1}[a4paper,border shrink=5mm]
}
\mode<presentation>{
  \usetheme{Juelich}
}

\usepackage{babel}
\usepackage[scaled]{helvet}
\usepackage{graphics}

\usepackage[utf8]{inputenc}
\usepackage[T1]{fontenc}

\usepackage{hyperref}

\title{Git-ify Your (digital) Life}
\subtitle{Git-basierte Tools, die das Leben vereinfachen}
\author{Torbjörn Klatt}
\institute{JSC}
\date{\today}


\begin{document}
\maketitle

\begin{frame}
  \frametitle{Overview}
  \begin{itemize}
    \item \hyperlink{git}{\textbf{Git} -- \textit{Eine Kurzübersicht}}
    \item \hyperlink{etckeeper}{\textbf{etckeeper} -- \textit{Behalte deine System-Configs}}
    \item \hyperlink{vcsh}{\textbf{vcsh} -- \textit{Versioniere dein \$HOME}}
    \item \hyperlink{mr}{\textbf{mr} -- \textit{Viele Repos}}
    \item \hyperlink{gitannex}{\textbf{git-annex} -- \textit{Git für Metadaten}}
    \item \hyperlink{bup}{\textbf{bup} -- \textit{Backup mit Git}}
    \item \hyperlink{ikiwiki}{\textbf{ikiwiki} -- \textit{Docu-Wiki als Git-Repo}}
    \item \hyperlink{tipps}{Tipps}
  \end{itemize}
\end{frame}


%%%%%%%%%%%%%%%%%%%%%%%%%%%%%%%%%%%%%%%%%%%%%%%%%%%%%%%%%%%%%%%%%%%%%%%%%%%%%%%%%%%%%%%%%%%%%%%%%%%%
\part{Git}
\makepart

\begin{frame}[label=git]
  \frametitle{Das Versions-Verwaltungssystem \textit{Git}}
  \framesubtitle{Eine kurze Übersicht}
  \begin{itemize}
    \item dezentral\\
      {\scriptsize wie Mercurial/hg, Bazar und im Gegensatz zu CVS, Subversion}
    \item von Linus ``the-one-and-only'' Torwalds\\
      {\scriptsize\textit{Git} im britisch-englischen Slang gleichbedeutend mit ``\textit{unpleasant person}''}
    \item nicht-linear\\
      {\scriptsize ``\textit{branching}'' und ``\textit{merging}'' einfach und performant}
    \item kryptographische Verifizierung von Revisionen\\
      {\scriptsize jede Revision (``\textit{commit}'') hat einen eindeutigen SHA-1-Hash}
  \end{itemize}
\end{frame}



%%%%%%%%%%%%%%%%%%%%%%%%%%%%%%%%%%%%%%%%%%%%%%%%%%%%%%%%%%%%%%%%%%%%%%%%%%%%%%%%%%%%%%%%%%%%%%%%%%%%
\part{etckeeper}
\makepart

\begin{frame}[label=etckeeper]
  \frametitle{etckepper}
  \framesubtitle{Behalte deine System-Configs \footnote{\tiny Windows-Nutzer: Lehnt euch zurück oder schlaft}}
  \begin{itemize}
    \item 
  \end{itemize}
\end{frame}


%%%%%%%%%%%%%%%%%%%%%%%%%%%%%%%%%%%%%%%%%%%%%%%%%%%%%%%%%%%%%%%%%%%%%%%%%%%%%%%%%%%%%%%%%%%%%%%%%%%%
\part{vcsh}
\makepart

\begin{frame}[label=vcsh]
  \frametitle{}
  \framesubtitle{}
  
\end{frame}


%%%%%%%%%%%%%%%%%%%%%%%%%%%%%%%%%%%%%%%%%%%%%%%%%%%%%%%%%%%%%%%%%%%%%%%%%%%%%%%%%%%%%%%%%%%%%%%%%%%%
\part{mr}
\makepart

\begin{frame}[label=mr]
  \frametitle{}
  \framesubtitle{}
  
\end{frame}


%%%%%%%%%%%%%%%%%%%%%%%%%%%%%%%%%%%%%%%%%%%%%%%%%%%%%%%%%%%%%%%%%%%%%%%%%%%%%%%%%%%%%%%%%%%%%%%%%%%%
\part{git-annex}
\makepart

\begin{frame}[label=gitannex]
  \frametitle{git-annex}
  \framesubtitle{Git für Metadaten}
  \begin{itemize}
    \item Problem vieler großer/selten genutzter Daten
    \item \textit{git-annex} speicher Name, Größe und Ort von Daten ohne sich lokal haben zu müssen
    \item entfernte Repositories können u.a. USB-HDDs/Sticks, WebDAV, rsync-Verzeichnisse oder das Web sein
    \item integrierter Podcast-Feed-Reader
  \end{itemize}
\end{frame}


%%%%%%%%%%%%%%%%%%%%%%%%%%%%%%%%%%%%%%%%%%%%%%%%%%%%%%%%%%%%%%%%%%%%%%%%%%%%%%%%%%%%%%%%%%%%%%%%%%%%
\part{bup}
\makepart

\begin{frame}[label=bup]
  \frametitle{bup}
  \framesubtitle{Git für große Dateien}
  \begin{itemize}
    \item Git ist für Quelltexte gebaut worden\\
      {\scriptsize Binärdateien sind für Git ein großer Blob}
    \item bup 
  \end{itemize}
\end{frame}


%%%%%%%%%%%%%%%%%%%%%%%%%%%%%%%%%%%%%%%%%%%%%%%%%%%%%%%%%%%%%%%%%%%%%%%%%%%%%%%%%%%%%%%%%%%%%%%%%%%%
\part{ikiwiki}
\makepart

\begin{frame}[label=ikiwiki]
  \frametitle{}
  \framesubtitle{}
  
\end{frame}


%%%%%%%%%%%%%%%%%%%%%%%%%%%%%%%%%%%%%%%%%%%%%%%%%%%%%%%%%%%%%%%%%%%%%%%%%%%%%%%%%%%%%%%%%%%%%%%%%%%%
\part{Tipps}
\makepart

\begin{frame}[label=tipps]
  \frametitle{Tipps}
  \framesubtitle{}
  \begin{itemize}
    \item Zsh als primäre Shell\\
      {\scriptsize built-in Vervollständigung aller Git-Befehle (inkl. Auswahl von Branches und Tags)}
    \item packt eure \LaTeX-Dokumente unter Git
  \end{itemize}
\end{frame}


%%%%%%%%%%%%%%%%%%%%%%%%%%%%%%%%%%%%%%%%%%%%%%%%%%%%%%%%%%%%%%%%%%%%%%%%%%%%%%%%%%%%%%%%%%%%%%%%%%%%
\begin{frame}
  \frametitle{Quellen}
  \begin{itemize}
    \item Stark inspiriert von Richard ``RichiH'' Hartmans Vortrag auf Linuxtag 2013
      \footnote{\tiny\url{http://www.linuxtag.org/2013/de/program/mittwoch-22-mai-2013.html?eventid=147}}
    \item offizielle und inoffizielle Dokumentation der genannten Tools
    \item (Langzeit-)Experimente mit den genannten Tools
  \end{itemize}
\end{frame}


\end{document}

