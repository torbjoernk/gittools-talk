\documentclass[english,hyperref={pdfpagelabels=false},aspectratio=1610]{beamer}

\usepackage{lmodern}
\usepackage[utf8]{inputenc}
\usepackage[T1]{fontenc}

\mode<handout>{
  \usepackage{pgf}
  \usepackage{pgfpages}
  \pgfpagesuselayout{2 on 1}[a4paper,border shrink=5mm]
}
\mode<presentation>{
  \usetheme{Juelich}
}

\usepackage[english]{babel}
\usepackage[scaled]{helvet}
\usepackage{graphics}

\usepackage{listings}

\usepackage{hyperref}

\lstdefinelanguage{zsh}{%
  morekeywords=[1]{cd,grep,du,bup,head,cat,git,vcsh,mr,ls,gpg,gitodo,etckeeper},
  morekeywords=[2]{annex,index,save,init,add,diff,fetch,commit,checkout,remote,push,clone,list,update,bootstrap,new},
  sensitive=true,
  morecomment=[l]\#,
  morecomment=[l]\>,
  string=[d]\"
}
\lstset{
%   language=zsh,
  basicstyle=\scriptsize\ttfamily,
  commentstyle=\color{fzjgray80},
  keywordstyle=[1]\color{fzjgreen},
  keywordstyle=[2]\color{fzjblue},
  stringstyle=\color{fzjred},
  inputencoding=utf8,
  showspaces=false,
  showstringspaces=false
}

\title{Git-based tools to ease your life}
\subtitle{Git-ify Your (digital) Life}
\author{Torbjörn Klatt <\href{mailto:t.klatt@fz-juelich.de}{t.klatt@fz-juelich.de}>}
\institute{JSC Internal Seminar}
\date{May 7, 2014}
\setbeamertemplate{footer element3}{PGP-Key: 0x64216AF3}
\setbeamertemplate{slide counter}[showall]


\begin{document}
\maketitle

\begin{frame}
  \frametitle{Overview}
  \begin{description}
    \item[\hyperlink{git}{Git}] \hyperlink{git}{\textit{a short review}}
    \item[\hyperlink{etckeeper}{etckeeper}] \hyperlink{etckeeper}{\textit{keep your system's configs}}
    \item[\hyperlink{vcsh}{vcsh}] \hyperlink{vcsh}{\textit{version your \$HOME}}
    \item[\hyperlink{mr}{mr}] \hyperlink{mr}{\textit{my / multiple repositories}}
    \item[\hyperlink{gitannex}{git-annex}] \hyperlink{gitannex}{\textit{so meta!}}
    \item[\hyperlink{bup}{bup}] \hyperlink{bup}{\textit{backup with Git}}
    \item[\hyperlink{gcrypt}{gcrypt}] \hyperlink{gcrypt}{\textit{GPG-encrypted Git repositories}}
  \end{description}
  \vfill
  \bigskip
  \vfill
  \scriptsize
  This talk is heavily inspired by Richard ``\textit{RichiH}'' Hartman's talk at \textit{Linuxtag 2013}
  \footnote{\tiny\url{http://www.linuxtag.org/2013/de/program/mittwoch-22-mai-2013.html?eventid=147}}, the official and unofficial documentation of named tools and (long-term) experiments with named tools.
\end{frame}


%%%%%%%%%%%%%%%%%%%%%%%%%%%%%%%%%%%%%%%%%%%%%%%%%%%%%%%%%%%%%%%%%%%%%%%%%%%%%%%%%%%%%%%%%%%%%%%%%%%%
\part{Git}
\makepart

\begin{frame}[label=git]
  \frametitle{Version Control System \textit{Git}}
  \framesubtitle{main features used by following tools}
  \begin{itemize}
    \item decentralized / distributed\\
      {\scriptsize alike Mercurial/hg or Bazar in contrast to CVS or Subversion}
    \item works on deltas {\scriptsize\color{fzjgray50}(diffs, patches)} instead of whole files
    \item non-linear history
    \item cryptological verification of revisions\\
      {\scriptsize each revision (\textit{commit}) has a unique SHA-1 hash computed from diff + meta info}
    \item no need for a server / everything is locally available\\
      {\scriptsize because of first point}
  \end{itemize}
\end{frame}



%%%%%%%%%%%%%%%%%%%%%%%%%%%%%%%%%%%%%%%%%%%%%%%%%%%%%%%%%%%%%%%%%%%%%%%%%%%%%%%%%%%%%%%%%%%%%%%%%%%%
\part{etckeeper}
\makepart

\begin{frame}[label=etckeeper]
  \frametitle{etckeeper -- Keep Your System's Configurations}
  \framesubtitle{Easy Backups for Your \texttt{/etc}~\footnote{\tiny by Joey Hess and others}}
  \begin{itemize}
    \item creates a Git {\scriptsize\color{fzjgray50}(or Mercurial/Bazaar/Darcs)} repo for \texttt{/etc}
    \item uses additional meta-file for remembering permissions for each file\\
      {\scriptsize DVCSs usually don't track file owner info; only executable bit}
    \item uses \textit{pre-commit} and \textit{post-update} hooks to fix file permissions
    \item hooks itself into package managers {\scriptsize\color{fzjgray50}(e.g. \textit{apt}, \textit{zypper})} to auto-commit \texttt{/etc} \\
      before and after package changes
    \item manual commits also possible
  \end{itemize}
\end{frame}

\begin{frame}[fragile]
  \frametitle{etckeeper -- Keep Your System's Configurations}
  \framesubtitle{\alt<1>{Example 1}{Examples 1 \& 2}}
  \begin{block}{Initialization}<1->
    \vspace{-0.75em}
    \begin{lstlisting}[language=zsh]
etckeeper init
# after some time
cd /etc && git log --oneline
> 5bb2977 daily autocommit
> cdd9c8c yast update
> 9b76558 I added some cron jobs
> 711446f initial commit
    \end{lstlisting}
    \vspace{-0.75em}
  \end{block}
  \begin{block}{Switching setup}<2>
    \vspace{-0.75em}
    \begin{lstlisting}[language=zsh]
# on April first
git checkout april_first_joke_etc
etckeeper init
# day later
git checkout master
etckeeper init
    \end{lstlisting}
    \vspace{-0.75em}
  \end{block}
\end{frame}

\begin{frame}[fragile]
  \frametitle{etckeeper -- Keep Your System's Configurations}
  \framesubtitle{Example 3}
  \begin{block}{Get difference between two system's configs}
    \vspace{-0.75em}
    \begin{lstlisting}[language=zsh]
git remote add my-other-host ssh://my-other-host/etc
git fetch my-other-host
git diff my-other-host/master group | head
> diff --git a/group b/group
> index 0242b84..b5e4384 100644
> --- a/group
> +++ b/group
> @@ -5,21 +5,21 @@ sys:x:3:
> adm:x:4:joey
> tty:x:5:
> disk:x:6:
> -lp:x:7:cupsys
> +lp:x:7:
    \end{lstlisting}
    \vspace{-0.75em}
  \end{block}
\end{frame}


%%%%%%%%%%%%%%%%%%%%%%%%%%%%%%%%%%%%%%%%%%%%%%%%%%%%%%%%%%%%%%%%%%%%%%%%%%%%%%%%%%%%%%%%%%%%%%%%%%%%
\part{vcsh}
\makepart

\begin{frame}[label=vcsh]
  \frametitle{vcsh -- Version Control System for (your) \$HOME}
  \framesubtitle{version \texttt{.profile}, \texttt{.\{bash,zsh,vim\}rc}, \dots~--- without pollution~\footnote{\tiny by Richard Hartmann and others}}
  \begin{itemize}
    \item separate Git repositories for dotfiles\\
      {\scriptsize without polluting \$HOME with \texttt{.git} directories}
    \item easily migrate your personalized environment to other hosts\\
      {\scriptsize clone your \texttt{.vim} repository on new host to have it synchronized}
    \item allows for different branches for different hosts\\
      {\scriptsize e.g. ``\texttt{tklatt-zamws}'', ``\texttt{myself-laptop}'', ``\texttt{su-myserver}''}
    \item \textit{vcsh} is a single Shell script
  \end{itemize}
\end{frame}

\begin{frame}[fragile]
  \frametitle{vcsh -- Version Control System for (your) \$HOME}
  \framesubtitle{Examples}
  \begin{block}{One repository for your Vim config \dots}<1->
    \vspace{-0.75em}
    \begin{lstlisting}[language=zsh]
vcsh init vim
vcsh vim add ~/.vimrc ~/.vim
vcsh vim commit -m "Initial commit of my Vim configuration"
vcsh vim remote add origin git@my-server.net:vim-repo
vcsh vim push -u origin master
    \end{lstlisting}
    \vspace{-0.75em}
  \end{block}
  \begin{block}{\dots another for Zsh}<2>
    \vspace{-0.75em}
    \begin{lstlisting}[language=zsh]
vcsh init zsh
vcsh zsh add ~/.zsh ~/.zshrc ~/.zshenv
vcsh zsh commit -m "Initial commit of my Zsh configuration"
vcsh zsh remote add origin git@my-server.net:zsh-repo
vcsh zsh push -u origin master
    \end{lstlisting}
    \vspace{-0.75em}
  \end{block}
\end{frame}


%%%%%%%%%%%%%%%%%%%%%%%%%%%%%%%%%%%%%%%%%%%%%%%%%%%%%%%%%%%%%%%%%%%%%%%%%%%%%%%%%%%%%%%%%%%%%%%%%%%%
\part{mr}
\makepart

\begin{frame}[label=mr,fragile]
  \frametitle{mr -- my / multiple repositories}
  \framesubtitle{One command to rule them all~\footnote{\tiny by Joey Hess and others}}
  \begin{itemize}
    \item Problem: a bunch of \textit{vcsh} repos are not very handy
    \item iterates over list of repos and runs same command on each
    \item can handle \emph{Git}, \emph{git-svn} and \emph{vcsh} repos equally
    \item provides \texttt{bootstrap} command to setup/clone an environment on new host
    \item integrates well with \textit{vcsh} {\scriptsize\color{fzjgray50}(\textit{mr} config directory can be a \textit{vcsh} repo itself)}
    \item a single Perl script
  \end{itemize}
  \begin{block}{Example 1: Cloning and Bootstrapping}
    \vspace{-0.75em}
    \begin{lstlisting}[language=zsh]
vcsh list
> vim zsh git ssh bin
mr update    # runs `git pull` or `git clone` for each
# downloads named .mrconfig and clones all repos in there
mr bootstrap https://my-server.net/.mrconfig
    \end{lstlisting}
    \vspace{-0.75em}
  \end{block}
\end{frame}

\begin{frame}[fragile]
  \frametitle{mr -- my / multiple repositories}
  \framesubtitle{Examples}
  \begin{block}{Status, Commit \& Push}
    \vspace{-0.75em}
    \begin{lstlisting}[language=zsh]
mr status
> mr status: /home/t.klatt/.config/vcsh/repo.d/git.git
> mr status: /home/t.klatt/.config/vcsh/repo.d/mr.git
> mr status: /home/t.klatt/.config/vcsh/repo.d/ssh.git
> mr status: /home/t.klatt/.config/vcsh/repo.d/vim.git
>  M .vim/bundles.vim
> mr status: /home/t.klatt/.config/vcsh/repo.d/zsh.git
>  M .zshrc
> mr status: finished (5 ok)

mr commit
# `git commit -a` is called on every dirty repo

mr push
# `git push` is called on every repo
    \end{lstlisting}
    \vspace{-0.75em}
  \end{block}
\end{frame}


%%%%%%%%%%%%%%%%%%%%%%%%%%%%%%%%%%%%%%%%%%%%%%%%%%%%%%%%%%%%%%%%%%%%%%%%%%%%%%%%%%%%%%%%%%%%%%%%%%%%
\part{git-annex}
\makepart

\begin{frame}[label=gitannex]
  \frametitle{git-annex -- Version Files Without Their Contents}
  \framesubtitle{So meta!~\footnote{\tiny by Joey Hess and others}}
  \begin{itemize}
    \item saves meta info {\scriptsize\color{fzjgray50}(i.e. name, size, SHA-1)} of files without their contents
    \item saves actual files read-only in \texttt{.git/annex/objects}\\
      {\scriptsize symlinks them to original/real location}
    \item keeps track of which remote has which files\\
      {\scriptsize each remote identified by UUID}
    \item designed for flaky connections\\
      {\scriptsize uses \emph{rsync} for data transfer}
  \end{itemize}
\end{frame}

\begin{frame}
  \frametitle{git-annex -- Version Files Without Their Contents}
  \framesubtitle{I mean, so really meta!}
  \begin{itemize}
    \item written in Haskell\\
      {\scriptsize available for Linux, OSX, Android (beta), Windows (alpha)}
    \item allows for special remotes
      \begin{itemize}
        \item Amazon S3 / Glacier
        \item WebDAV
        \item rsync
        \item the web {\scriptsize\color{fzjgray50}(\texttt{http(s)://}, \texttt{ftp://}, archive.org, arxiv.org/[format]/[ID], etc.)}
        \item podcast feeds
        \item XMPP
        \item simple directories
        \item Google Drive, OwnCloud, IMAP, Flickr, \dots you name it \dots
      \end{itemize}
    \item example collection of some conference proceedings {\scriptsize\color{fzjgray50}(slides + video recordings)}\\
      {\scriptsize \url{https://github.com/RichiH/conference_proceedings}}
  \end{itemize}
\end{frame}

\begin{frame}[fragile]
  \frametitle{git-annex -- Version Files Without Their Contents}
  \framesubtitle{Example Szenario: The Archivist \footnote{\tiny taken from official website}}
  \begin{itemize}
    \item annex all files
    \item actual files offline in special remotes on USB drives, tapes, etc.
    \item having full info about name, size and location of all files in one place at hand
  \end{itemize}
  
  \begin{block}{Example}
    \vspace{-0.75em}
    \begin{lstlisting}[language=zsh]
git annex whereis
> whereis my_cool_big_file (1 copy)
>     7570b02e-15e9-11e0-adf0-9f3f94cb2eaa  -- backup drive
> whereis other_file (3 copies)
>     0c443de8-e644-11df-acbf-f7cd7ca6210d  -- here (laptop)
>     62b39bbe-4149-11e0-af01-bb89245a1e61  -- usb drive
>     7570b02e-15e9-11e0-adf0-9f3f94cb2eaa  -- backup drive
    \end{lstlisting}
    \vspace{-0.75em}
  \end{block}
\end{frame}



%%%%%%%%%%%%%%%%%%%%%%%%%%%%%%%%%%%%%%%%%%%%%%%%%%%%%%%%%%%%%%%%%%%%%%%%%%%%%%%%%%%%%%%%%%%%%%%%%%%%
\part{bup}
\makepart

\begin{frame}[label=bup]
  \frametitle{bup -- Git for \textsc{Large} Files}
  \framesubtitle{\dots and binary data!~\footnote{\tiny by over 40 people}}
  \begin{itemize}
    \item recap: Git is designed for plaintext files\\
      {\scriptsize binary files are just a huge blob for Git; no diff possible}
    \item uses Git object trees and replaces hashing and packing algorithms
    \item designed for space-saving incremental backups
    \item backups can be FUSE mounted
    \item can be a special remote for \textit{git-annex}
    \item \texttt{bup web}: browse backup trees in web browser
    \item written in Python
  \end{itemize}
\end{frame}

\begin{frame}[fragile]
  \frametitle{bup -- Git for \textsc{Large} Files}
  \framesubtitle{Examples}
  \begin{block}{Initialization}<1->
    \vspace{-0.75em}
    \begin{lstlisting}[language=zsh]
bup init
> Initialized empty Git repository in /backup/homebck
bup index /home/t.klatt                                 # took a few seconds
bup save --name zamws-home-tklatt /home/t.klatt         # took about 3min
> Reading index: 203502, done.
> Saving: 100.00% (14743111/14743111k, 203502/203502 files), done.
    \end{lstlisting}
    \vspace{-0.75em}
  \end{block}
  \begin{block}{Browsing}<2>
    \vspace{-0.75em}
    \begin{lstlisting}[language=zsh]
# after some time (and daily cron job for backup)
bup ls zamws-home-tklatt | grep -ce '^2014-.*'
> 71
du -sh /home/t.klatt $BUP_DIR
> 18G    /home/t.klatt
> 24G    /backup/homebck
    \end{lstlisting}
    \vspace{-0.75em}
  \end{block}
\end{frame}

{
\setbeamertemplate{background canvas}{\includegraphics[width=\paperwidth]{black.png}}
\begin{frame}[plain]
  \centering
  \color{white}
  {\Huge \textsc{The Crash}}
  \vfill
  {\scriptsize \emph{the cat running over the keyboard or a stupid action} {\color{fzjgray50}(\texttt{cd \textasciitilde~\&\& rm -rf *})} \emph{while being sleep-deprived}}
  \vfill
  \color{green}
  \texttt{[t.klatt@zam]>\textunderscore}
\end{frame}
}

\begin{frame}[fragile]
  \frametitle{bup -- Git for \textsc{Large} Files}
  \framesubtitle{Example}
  \begin{block}{Backup Recovering}
    \vspace{-0.75em}
    \begin{lstlisting}[language=zsh]
cd /home/t.klatt
bup restore -C . zamws-home-tklatt/latest/home/t.klatt/
> Restoring ...  # takes some time;
                 # a little longer than `bup init & bup save`
    \end{lstlisting}
    \vspace{-0.75em}
  \end{block}
\end{frame}



%%%%%%%%%%%%%%%%%%%%%%%%%%%%%%%%%%%%%%%%%%%%%%%%%%%%%%%%%%%%%%%%%%%%%%%%%%%%%%%%%%%%%%%%%%%%%%%%%%%%
\part{gcrypt}
\makepart

\begin{frame}[label=gcrypt]
  \frametitle{gcrypt -- GPG-encrypted Git remotes}
  \framesubtitle{}
  \begin{itemize}
    \item implements a git-remote-handler to deal with \texttt{gcrypt::} remotes\\
      {\scriptsize transport via \emph{rsync}, \emph{sftp} or \emph{git} protocols}
    \item remote repository is GPG-encrypted for one or multiple participants
    \item each pack is encrypted with a symmetric key stored in a asymmetric encrypted manifest file
    \item can be a special remote for \textit{git-annex}
    \item Hint: use it as a remote for your \textit{etckeeper}'s repo
    \item \scriptsize\color{fzjgray50}Remark: You might want to use Joey ``joeyh'' Hess' fork of \emph{gcrypt}
      \footnote{\tiny\url{https://github.com/joeyh/git-remote-gcrypt} because it has some bugfixes not yet merged upstream}
  \end{itemize}
\end{frame}

\begin{frame}[fragile]
  \frametitle{gcrypt -- GPG-encrypted Git remotes}
  \framesubtitle{Example}
  \begin{block}{Initialization and Committing}
    \vspace{-0.75em}
    \begin{lstlisting}[language=zsh]
git add my_secret_file & git commit -m "secret file"
git remote add secret-server gcrypt::git@my-server.net:secret-repo
git push secret-server master
    \end{lstlisting}
    \vspace{-0.75em}
  \end{block}
  \begin{block}{Cloning as Another Person}
    \vspace{-0.75em}
    \begin{lstlisting}[language=zsh]
git clone git@my-server.net:secret-repo
ls -lA secret-repo
> -rw------- 1 t.klatt users  303 Jan 15 09:24 0153f2b0...ea5f861d
> -rw------- 1 t.klatt users 1.4K Jan 15 09:24 91bd0c09...4881aa0a
> drwx------ 1 t.klatt users  138 Jan 15 09:25 .git
gpg -d 91bd0c09...4881aa0a
> fc564bef...94c3ff80 refs/heads/master
> pack :SHA256:0153f2b0...ea5f861d w+bxes2v...1MCkGi8+
> repo :id:3lmzxTGoXJVmHPtfaOTf
    \end{lstlisting}
    \vspace{-0.75em}
  \end{block}
\end{frame}



%%%%%%%%%%%%%%%%%%%%%%%%%%%%%%%%%%%%%%%%%%%%%%%%%%%%%%%%%%%%%%%%%%%%%%%%%%%%%%%%%%%%%%%%%%%%%%%%%%%%
\begin{frame}
  \frametitle{Project Links}
  \begin{description}
    \item[etckeeper] \url{https://github.com/joeyh/etckeeper}
    \item[vcsh] \url{https://github.com/RichiH/vcsh}
    \item[mr] \url{https://github.com/joeyh/myrepos}
    \item[git-annex] \url{https://git-annex.branchable.com/}
    \item[bup] \url{https://github.com/bup/bup}
    \item[gcrypt] \url{https://github.com/blake2-ppc/git-remote-gcrypt}\\
      {\tiny Joey Hess' fork with bug fixes: \url{https://github.com/joeyh/git-remote-gcrypt}}
  \end{description}
\end{frame}

\begin{frame}
  \frametitle{~}
  \begin{center}
    {\huge Thank you for your interest and time!}\par
    \bigskip
    \bigskip
    \bigskip
    {\Large Questions?}\par
    {\scriptsize\color{fzjgray50}(now or later)}\par
    \bigskip
    \bigskip
    \begin{columns}
      \tiny
      \begin{column}{0.1\textwidth}
      \end{column}
      \begin{column}{0.4\textwidth}
        \begin{description}
          \item[Building] 16.3
          \item[Room] 022
          \item[Tel] 96452
          \item[eMail] \href{mailto:t.klatt@fz-juelich.de}{t.klatt@fz-juelich.de}
        \end{description}
      \end{column}
      \begin{column}{0.38\textwidth}
        \begin{description}
          \item[PGP-Key] \texttt{0x64216AF3}
          \item[Fingerprint] \texttt{7176 4979 01E4 C412 BCCC B403 F4E5 CF72 6421 6AF3}
        \end{description}
      \end{column}
      \begin{column}{0.12\textwidth}
      \end{column}
    \end{columns}
  \end{center}
\end{frame}



\end{document}

